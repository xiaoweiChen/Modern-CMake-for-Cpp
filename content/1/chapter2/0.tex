Writing in the CMake Language is a bit tricky. When you read a CMake listfile for
the first time, you may be under the impression that the language in it is so simple
that it doesn't require any special training or preparation. What follows is very often
a practical attempt to introduce changes and experiment with the code without
a thorough understanding of how it works. We programmers are usually very busy and
are overly keen to tackle any build-related issues with little investment. We tend to make
gut-based changes hoping they just might do the trick. This approach to solving technical
problems is called voodoo programming.

The CMake Language appears simple: after we have completed our small addition, fix,
or hack, or added a one-liner, we realize that something isn't working. The time spent on
debugging is often longer than that spent on actually studying the subject. Luckily, this
won't be our fate – because this chapter covers the vast majority of the critical knowledge
needed to use the CMake Language in practice.

In this chapter, we'll not only learn about the building blocks of the CMake Language – comments, commands, variables, and control structures – but we'll also give the necessary background and try them out in a clean and modern CMake example. CMake puts you in a bit of a unique position. On one hand, you perform a role of a build engineer; you need to understand all the intricacies of the compilers, the platforms, and everything else in-between. On the other hand, you're a developer; you're writing code that generates a buildsystem. Writing good code is hard and requires thinking on many levels at the same time – it should work and be easy to read, but it should also be easy to analyze, extend, and maintain. This is exactly what we're going to talk about here.

Lastly, we'll introduce some of the most useful and common commands in CMake. Commands that aren't used that often will be placed in the Appendix section (this will include a complete reference guide for the string, list, and file manipulation commands).

In this chapter, we're going to cover the following main topics:

\begin{itemize}
\item 
The basics of the CMake Language syntax

\item 
Working with variables

\item 
Using lists

\item 
Understanding control structures in CMake

\item 
Useful commands
\end{itemize}