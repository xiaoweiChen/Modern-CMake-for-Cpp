将源码转换为可工作应用会比较神奇。不仅是效果本身(即设计并赋予生命的工作机制),而且是将理念付诸于过程的行为本身。

作为开发者的工作流程基本是:设计、编码和测试。我们使用编译器能理解的语言来编写代码,并检查写出来的东西是否如预期的那样工作。为了从源码中创建高质量的应用,需要耐心地重复流程,并检查容易出错的任务:使用正确的命令、检查语法、链接二进制文件、运行测试、产生报告问题等。

每一次的进步都不容易,但我们更想把重点放在编码上,并将其他所有事情委托给自动化工具。理想情况下,这个过程应该从一个简单地按钮开始。在更改了代码之后,工具将以智能的、快速的、可扩展的方式,在不同的操作系统和环境中以相同的方式工作。支持多个集成开发环境(IDE)和持续集成(CI)流水,这些流水在更改提交到代码库后,就能对代码进行测试。

CMake是许多此类需求一种答案,并且要正确配置和使用需要一定的工作量。这并不是因为CMake复杂,而是因为需求很复杂。请放心,我们会有条不紊地进行学习。完全了解这个流程后,您将成为一个软件构建大师。

可能有的读者会急于开始编写自己的CMake项目,我对您的态度表示赞赏。由于您的项目将主要面向用户(包括您自己),因此理解如何使用同样重要。

那么先从成为CMake的用户开始吧。先介绍一些基础知识:这个工具是什么,如何工作,如何安装。然后,将深入研究命令行和操作模式。最后,将总结项目中不同文件的用途,并了解如何在不创建项目的情况下使用CMake。

本章中,我们将讨论以下主题:

\begin{itemize}
\item 
基础知识

\item 
不同平台的安装

\item 
使用命令行

\item 
项目文件

\item 
脚本和模块
\end{itemize}

