现在开始讨论CMake的核心功能:构建项目。CMake中,一个项目包含所有必要的源文件和配置,以管理将解决方案变为现实的过程。

配置从执行所有检查开始:是否支持目标平台,是否具有所有必要的依赖项和工具,所提供的编译器是否工作并支持所需的特性。

完成后,CMake将选择的构建工具生成一个构建系统并运行它。源文件将编译并链接,以产生输出工件。

开发者可以在内部使用项目来生成包,用户可以通过包管理器将这些包安装到他们的系统上,也可以使用提供的单可执行安装程序。项目还可以在开源存储库中共享,这样用户就可以使用CMake在自己的机器上编译项目并安装。

充分利用CMake项目的潜力将改善开发体验和生成代码的质量,可以自动化许多枯燥的任务,例如在构建后运行测试,检查代码覆盖率,格式化代码,以及使用Linter和其他工具检查源码。

为了解锁CMake项目的强大功能,先讨论一些关键的决策——这些是如何将项目作为一个整体正确配置,如何对其进行分区和设置源树,以便所有文件都整齐地组织在正确的目录中。

然后学习如何查询构建项目的环境——例如,它是什么架构?可用的工具是什么?支持哪些特性?使用的是什么标准的语言?最后,将学习如何编译一个测试C++文件,以验证所选编译器是否满足项目的需求标准。

本章中,我们将讨论以下主题:

\begin{itemize}
\item 
指令和命令

\item 
划分项目

\item 
项目结构

\item 
环境范围

\item 
配置工具链

\item 
禁用内构建
\end{itemize}

