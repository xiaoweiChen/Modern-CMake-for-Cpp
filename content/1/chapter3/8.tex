本章中介绍了许多概念,这将为我们前进和构建坚固的、经得起考验的项目提供坚实的基础。我们讨论了如何设置CMake的最低版本以及如何配置项目的关键方面——即名称、语言和元数据字段。

打下良好的基础将有助于确保我们的项目能够快速发展,这就是项目划分的原因。使用include()分析了原生代码分区,并将其与add\_subdirectory()进行了比较。至此,了解了管理变量的目录作用域的好处,并探讨了使用更简单的路径和增加的模块化。当需要将代码分解成更独立的单元时,拥有创建嵌套项目并单独构建它的选项非常有用。

概述了可以使用的分区机制之后,探讨了如何使用——例如,如何创建透明的、弹性的和可扩展的项目结构。具体来说,分析了CMake如何遍历列表文件,以及不同配置步骤的正确顺序。

接下来,研究了如何确定目标计算机和主机计算机的环境范围、它们之间的区别,以及通过不同的查询可以获得关于平台和系统的哪些信息。

最后,了解了如何配置工具链——例如,如何指定所需的C++版本,如何解决特定于供应商的编译器扩展问题,以及如何启用重要的优化。最后,发现了如何测试编译器所需的特性和编译测试文件。

虽然这是一个项目在技术上需要的全部内容,但它仍然不是一个非常有用的项目,我们需要引入目标。到目前为止,已经在这里或那里提到了它们,但在了解了一些一般概念之前,我试图避开这个主题。现在完成了这些铺垫,接下来将详细研究“目标”。