The most basic target we can build in CMake is a single binary executable file that encompasses an entire application. It can be made out of a single piece of source code, such as the classic helloworld.cpp. Or it can be something complex – built with hundreds or even tens of thousands of files. This is what many beginner projects look like – create a binary with one source file, add another, and, before you know it, everything is linked to a single binary without any structure whatsoever.

As software developers, we deliberately draw boundaries and designate components to group one or more units of translation (.cpp files). We do it for multiple reasons: to increase code readability, manage coupling and connascence, speed up the build process, and finally, extract the reusable components. Every project that is big enough will push you to introduce some form of partitioning.

A target in CMake is an answer to exactly that problem – a high-level logical unit that forms a single objective for CMake. A target may depend on other targets, and they are produced in a declarative way. CMake will take care of determining in what order targets have to be built and then execute the necessary steps one by one. As a general rule, building a target will produce an artifact that will be fed into other targets or delivered as a final product of the build.

I deliberately use the inexact word artifact because CMake doesn't limit you to producing executables or libraries. In reality, we can use generated buildsystems to create many kinds of output: more source files, headers, object files, archives, and configuration files – anything really. All we need is a command-line tool (such as a compiler), optional input files, and an output path.

Targets are a very powerful concept and simplify building a project greatly. It is key to understand how they work and how to configure them in the most elegant and clean way.

In this chapter, we're going to cover the following main topics:

\begin{itemize}
\item 
The concept of a target

\item 
Writing custom commands

\item 
Understanding generator expressions
\end{itemize}