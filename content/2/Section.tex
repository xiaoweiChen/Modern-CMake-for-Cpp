Now that we have the most essential skills, it's time to start diving a little deeper. The next part will allow you to solve most situations that come your way when building a project in CMake.

We purposely focus on modern, elegant practices and avoid bringing too much legacy into the picture. Specifically, we'll be dealing with logical build targets rather than manipulating individual files.

Next, we'll explain in detail all the steps that the toolchain takes to build a binary artifact from a target. That's the part many books about C++ are missing: how to configure and use preprocessors, compilers, and linkers properly, as well as how to optimize their behavior.

Lastly, this section will cover all the different ways in which CMake offers to manage dependencies, and will explain how to pick the best one for your specific use case.

本节包括以下几章:

\begin{itemize}
\item Chapter 4, Working with Targets
\item Chapter 5, Compiling C++ Sources with CMake
\item Chapter 6, Linking with CMake
\item Chapter 7, Managing Dependencies with CMake
\end{itemize}