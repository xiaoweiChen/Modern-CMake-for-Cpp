
编译源代码之后,可能希望避免在同一个平台上再次编译,甚至尽可能地与外部项目共享。当然,可以简单地按照最初创建的方式提供所有的目标文件,但这有一些缺点。分发多个文件并将它们单独添加到构建系统比较困难。这可能是一件麻烦的事,尤其是当他们数量众多的时候。相反,可以简单地将所有对象文件放到一个对象中并共享。可以用一个简单的add\_library()指令创建这些库(与target\_link\_libraries()一起使用)。按照惯例,所有库都有共同的前缀lib,并使用特定于系统的扩展来表示它们是哪种类型的库:

\begin{itemize}
\item 
静态库在类Unix系统上的后缀为.a,在Windows上的后缀为.lib。

\item 
动态库在类Unix系统上的后缀为.so,在Windows上的后缀为.dll。
\end{itemize}

在构建库(静态、动态或模块)时,经常会遇到此过程的名称链接。甚至CMake在chapter06/01-libraries项目的构建输出中也这样称呼它:

\begin{tcblisting}{commandshell={}}
[ 33%] Linking CXX static library libmy_static.a
[ 66%] Linking CXX shared library libmy_shared.so
[100%] Linking CXX shared module libmy_module.so
[100%] Built target module_gui
\end{tcblisting}

与看起来不同的是,链接器并不是用来创建前面所有的库的。执行重定位和引用解析也有例外。来看看每种库类型,了解它们的工作原理。

\subsubsubsection{6.3.1\hspace{0.2cm}静态库}

要构建一个静态库,可以简单地使用我们在前几章中看到的指令:

\begin{lstlisting}[style=styleCMake]
add_library(<name> [<source>...])
\end{lstlisting}

若BUILD\_SHARED\_LIBS变量没有设置为ON,前面的代码将生成一个静态库。若想要构建一个静态库,可以提供一个关键字:

\begin{lstlisting}[style=styleCMake]
add_library(<name> STATIC [<source>...])
\end{lstlisting}

什么是静态库?本质上是存储在存档中的原始对象文件的集合。类Unix系统上,可以通过ar工具创建此类库。静态库是提供代码编译版本的最古老和最基本的机制。若想避免将依赖关系与可执行文件分离,就可以使用,因为可执行文件的大小和使用的内存会增加。

静态库可能包含一些额外的索引,以加快最终的链接过程。每个平台都使用自己的方法来生成这些内容。类Unix系统使用ranlib工具来实现。

\subsubsubsection{6.3.2\hspace{0.2cm}动态库}

可以用SHARED关键字构建动态库:

\begin{lstlisting}[style=styleCMake]
add_library(<name> SHARED [<source>...])
\end{lstlisting}

也可以通过设置BUILD\_SHARED\_LIBS变量为ON并使用短版本来实现:

\begin{lstlisting}[style=styleCMake]
add_library(<name> [<source>...])
\end{lstlisting}

与静态库的区别很明显。动态库是使用链接器构建的,将执行链接的两个阶段。所以将收到一个具有适当的区段头、区段和区段头表的文件(图6.1)。

动态库(也称为共享对象)可以在多个不同的应用程序之间共享。一个操作系统会将这样一个库的单个实例与第一个使用它的程序一起加载到内存中,所有随后启动的程序都将提供相同的地址(这要感谢虚拟内存的复杂机制)。只有.data和.bss段将分别为使用库的每个进程创建(以便每个进程可以修改自己的变量而不影响其他使用者)。

由于采用了这种方法,系统中的整体内存使用更好了。若正在使用一个非常流行的库,可能不需要将它与程序一起发布。它很可能已经在目标机器上可用。但若不是这样,则希望用户在运行应用程序之前显式地安装它。当库的安装版本与预期版本不同时,这就有可能出现一些问题(这种类型的问题称为依赖地狱,更多信息可以在扩展阅读部分找到)。

\subsubsubsection{6.3.3\hspace{0.2cm}模块库}

要构建模块库,需要使用MODULE关键字:

\begin{lstlisting}[style=styleCMake]
add_library(<name> MODULE [<source>...])
\end{lstlisting}

这是动态库的一种,目的是作为运行时加载的插件使用,而不是在编译过程中链接到可执行文件。共享模块不会在程序启动时自动加载(像普通的共享库一样)。只有程序通过LoadLibrary(Windows)或dlopen()/dlsym() (Linux/macOS)等系统调用显式地请求它时,才会使用的到。

不应该尝试将可执行文件与模块链接起来,因为这并不能保证在所有平台上都能工作。若需要这样做,请使用动态库。

\subsubsubsection{6.3.4\hspace{0.2cm}位置无关的代码}

所有共享库和模块的源代码都应该在编译时启用位置无关代码标志。CMake检查目标的POSITION\_INDEPENDENT\_CODE属性,并适当添加特定于编译器的编译标志,如gcc或clang的-fPIC。

PIC是一个有点令人困惑的术语。现在,程序在某种意义上已经是位置无关的,因为使用虚拟内存抽象出实际的物理地址。当使用函数时,CPU使用内存管理单元(MMU)将虚拟地址(每个进程从0开始)转换为分配时可用的物理地址。这些映射不必指向连续的物理地址或遵循任何其他特定的顺序。

PIC是关于将符号(对函数和全局变量的引用)映射到其运行时地址。编译库期间,不知道哪些进程可能会使用。不可能预先确定库将在虚拟内存中的什么位置或以什么顺序加载。所以符号的地址是未知的,与库机器码的相对位置也是未知的。

为了解决这个问题,需要另一种间接方式。PIC将向我们的输出添加一个新部分——全局偏移表(GOT)。最终,这个部分将成为包含共享库所需的所有符号的运行时地址的段。GOT相对于.text段的位置在链接过程中是已知的;因此,所有符号引用都可以(通过偏移量)指向当时的占位符get。只有在第一次执行访问引用符号的指令时,内存中指向符号的实际值才会填充。这时,加载器将在GOT中设置该特定条目(这就是术语延迟加载的来源)。

动态库和模块的POSITION\_INDEPENDENT\_CODE属性会自动设置为ON。然而,重要的是要记住,若动态库链接到另一个目标,例如静态库或对象库,那么也需要在该目标上设置此属性。方法如下:

\begin{lstlisting}[style=styleCMake]
set_target_properties(dependency_target
					PROPERTIES POSITION_INDEPENDENT_CODE
					ON)
\end{lstlisting}

若不这样做,就会在使用CMake时遇到麻烦,因为默认情况下,这个属性会按照第4章中描述的方式检查冲突。

说到符号,还有一个问题要讨论。下一节将讨论导致定义歧义和名称冲突。











