在第5章中讨论了C++程序的生命周期,由五个主要阶段组成——编写、编译、链接、加载和执行。正确编译所有源代码之后,需要将它们放到一个可执行文件中。编译过程中产生的目标文件不能由处理器直接执行,这是为什么呢?

为了回答这个问题,就来看看编译器如何用ELF格式(类Unix系统和许多其他系统使用)构造对象文件的:

\begin{center}
\includegraphics[width=0.8\textwidth]{content/2/chapter6/images/1.jpg}\\
图6.1 目标文件的结构
\end{center}

编译器将为每个翻译单元(为每个.cpp文件)准备一个目标文件。这些文件将用于构建程序的内存映像。Object文件包含以下元素:

\begin{itemize}
\item 
ELF头标识目标操作系统、ELF文件类型、目标指令集体系结构,以及ELF文件中两个头表的位置和大小信息——程序头表(不存在于目标文件中)和节头表。

\item 
包含按类型分组的信息的部分(下面将介绍)。

\item 
节头表,包含关于名称、类型、标志、内存中的目标地址、文件中的偏移量和其他杂项信息,用于了解文件中的哪些部分,以及它们在哪里,就像目录一样。
\end{itemize}

当编译器处理源代码时,将收集到的信息分组到几个单独的格中,这些格将放在属于它们自己的区段中。其中一些如下:

\begin{itemize}
\item 
.text区段:机器代码,包含处理器要执行的所有指令

\item 
.data区段:初始化的全局对象和静态对象(变量)的所有值

\item 
.bss区段:未初始化的全局对象和静态对象(变量)的所有值,这些值将在程序启动时初始化为零

\item 
.rodata区段:常量的所有值(只读数据)

\item 
.strtab区段: 一个字符串表,包含所有常量字符串,例如hello.cpp示例中的Hello World

\item 
.shstrtab区段: 包含所有部分名称的字符串表
\end{itemize}

这些区段非常类似于可执行文件的最终版本,其将放入RAM中以运行我们的应用程序,但不能直接将这个文件加载到内存中。因为每个目标文件都有自己的一组区段。若只是把它们连在一起,就会遇到各种各样的问题。将浪费大量的空间和时间,因为需要更多的RAM页,指令和数据将很难复制到CPU缓存中。整个系统必须要复杂得多,并且会浪费宝贵的CPU周期,在运行时跳过许多(可能是数万个).text、.data和其他部分。

我们要做的是把对象文件的每个部分,和所有其他对象文件中相同类型的部分放在一起。这个过程称为重定位(这就是ELF文件类型为目标文件重定位的原因)。除了将适当的部分组合在一起之外,还必须更新文件中的内部关联——即变量、函数、符号表索引或字符串表索引的地址。所有这些值都是对象文件的本地值,其编号从0开始。将文件捆绑在一起时,需要偏移这些值,使它们指向组合文件中的正确地址。

图6.2显示了重新定位的过程——重新定位.text,从所有链接的文件构建.data,然后是.rodata和.strtab(为了简单起见,图中不包含头文件):

\begin{center}
\includegraphics[width=0.8\textwidth]{content/2/chapter6/images/2.jpg}\\
图6.2 .data区段的重定位
\end{center}

其次,链接器需要解析引用。每当来自翻译单元的代码引用另一个翻译单元中定义的符号时(例如:通过包含其头或使用extern关键字),编译器读取该声明并相信定义就在某个地方,稍后将提供该定义。链接器负责收集这些对外部符号的未解析引用,查找并填充它们在合并到可执行文件后驻留的地址。图6.3显示了一个简单的引用示例:

\begin{center}
\includegraphics[width=0.8\textwidth]{content/2/chapter6/images/3.jpg}\\
图6.3 一个引用解析
\end{center}

若开发者不知道其是如何工作的,这部分链接可能是问题的来源。最终可能会得到无法解析的引用,这些引用找不到外部符号,或者恰恰相反——提供了太多的定义,链接器不知道选择哪一个。

最终的可执行文件看起来非常类似于目标文件,包含带有解析引用的重定位区段、区段头表,当然还有描述整个文件的ELF头。主要的区别是程序头的存在(如图6.4所示)。

\begin{center}
\includegraphics[width=0.8\textwidth]{content/2/chapter6/images/4.jpg}\\
图6.4  ELF中可执行文件的结构
\end{center}

程序头放置在ELF头的右边,系统加载器将读取此头文件以创建进程映像。其头部会包含一些通用信息和内存布局的描述。布局中的每个条目代表一个称为段的内存片段。条目指定将读取哪些节、以什么顺序、虚拟内存中的哪个地址、标志是什么(读、写或执行),以及其他一些有用的细节。

目标文件也可以捆绑在库中,库是一个中间产品,可以在最终的可执行文件或另一个库中使用。下一节中,我们将讨论三种类型的库。



