如何使项目A的目标对项目B可用?通常使用find\_package()指令,但这需要创建一个包并将其安装到系统上。这种方法没问题,但需要一些工作量。有时,只需要一种非常快速的方法来构建一个项目,并使其目标可用于其他项目。

可以通过包含A的主列表文件来节省一些时间:已经包含了所有的目标定义,还可能包含许多其他东西:全局配置、需求、带有副作用的CMake指令、附加依赖项,以及在B中不想要的目标(例如单元测试)。所以,不要这样做。最好是通过提供项目B可以使用include()包含的目标导出文件来进行:

\begin{lstlisting}[style=styleCMake]
cmake_minimum_required(VERSION 3.20.0)
project(B)
include(/path/to/project-A/ProjectATargets.cmake)
\end{lstlisting}

这样做将为A的所有目标提供定义(如add\_library()和add\_executable()等指令),并设置正确的属性。

当然,不需要手动编写这样的文件——不是非常DRY的方法。CMake可以用export()指令生成这些文件,该指令具有以下签名:

\begin{lstlisting}[style=styleCMake]
export(TARGETS [target1 [target2 [...]]]
	[NAMESPACE <namespace>] [APPEND] FILE <path>
	[EXPORT_LINK_INTERFACE_LIBRARIES])
\end{lstlisting}

必须在TARGET关键字之后提供想要导出的所有目标,并在FILE之后提供目标文件名。其他参数可选:

\begin{itemize}
\item 
建议使用NAMESPACE作为提示,说明目标已从其他项目导入。

\item 
APPEND告诉CMake,不应该在写入之前擦除文件的内容。

\item 
EXPORT\_LINK\_INTERFACE\_LIBRARIES将导出目标链接依赖项(包括导入的和配置特定的变量)。
\end{itemize}

用示例Calc库来看看它的作用,其提供了两个简单的方法:

\begin{lstlisting}[style=styleCXX]
// chapter-11/01-export/src/include/calc/calc.h

#pragma once
int Sum(int a, int b);
int Multiply(int a, int b);
\end{lstlisting}

我们这样声明其目标:

\begin{lstlisting}[style=styleCMake]
# chapter-11/01-export/src/CMakeLists.txt

add_library(calc STATIC calc.cpp)
target_include_directories(calc INTERFACE include)
\end{lstlisting}

然后,用export(TARGETS)指令生成导出文件:

\begin{lstlisting}[style=styleCMake]
# chapter-11/01-export/CMakeLists.txt (fragment)

cmake_minimum_required(VERSION 3.20.0)
project(ExportCalcCXX)
add_subdirectory(src bin)
set(EXPORT_DIR "${CMAKE_CURRENT_BINARY_DIR}/cmake")
export(TARGETS calc
	FILE "${EXPORT_DIR}/CalcTargets.cmake"
	NAMESPACE Calc::
)
...
\end{lstlisting}

可以看到EXPORT\_DIR变量设置为构建树的cmake子目录(根据.cmake文件的约定)。导出目标声明文件CalcTargets.cmake,对于包含此文件的项目,有一个可见的calc::calc目标calc。

注意,这个导出文件还不是一个包,这个文件中的所有路径都是绝对的,并且硬编码到构建树中,所以它们不可重定位。

export()指令也有一个简短的版本:

\begin{lstlisting}[style=styleCMake]
export(EXPORT <export> [NAMESPACE <namespace>] [FILE
	<path>])
\end{lstlisting}

但需要<export>名称,而不是导出目标的列表。这样的<export>实例是由install(targets)定义的目标的命名列表。这里有一个例子,展示了这种简版在实践中如何使用:

\begin{lstlisting}[style=styleCMake]
# chapter-11/01-export/CMakeLists.txt (continued)

...
install(TARGETS calc EXPORT CalcTargets)
export(EXPORT CalcTargets
	FILE "${EXPORT_DIR}/CalcTargets2.cmake"
	NAMESPACE Calc::
)
\end{lstlisting}

现在,export()和install()指令之间共享一个目标列表。

两种生成导出文件的方法将产生相同的结果,将包含一些样板代码和几行定义目标的代码。将/tmp/b设置为构建树路径:

\begin{lstlisting}[style=styleCMake]
# /tmp/b/cmake/CalcTargets.cmake (fragment)

# Create imported target Calc::calc
add_library(Calc::calc STATIC IMPORTED)
set_target_properties(Calc::calc PROPERTIES
	INTERFACE_INCLUDE_DIRECTORIES
	"/root/examples/chapter11/01-export/src/include"
)
# Import target "Calc::calc" for configuration ""
set_property(TARGET Calc::calc APPEND PROPERTY
	IMPORTED_CONFIGURATIONS NOCONFIG
)
set_target_properties(Calc::calc PROPERTIES
	IMPORTED_LINK_INTERFACE_LANGUAGES_NOCONFIG "CXX"
	IMPORTED_LOCATION_NOCONFIG "/tmp/b/libcalc.a"
)
\end{lstlisting}

通常,我们不会编辑这个文件,甚至不会打开它,这里想突出显示这个生成文件中的硬编码路径。在当前的形式中,包不可重定位。若想改变这一点,需要先克服一些困难。我们将在下一节中探讨这为什么很重要。












