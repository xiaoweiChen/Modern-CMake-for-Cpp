从源码构建项目有它的好处,但会花费很长时间。对于只想使用包的最终用户来说,这不是最好的体验,特别是不是开发人员的情况下。软件分发的一种更方便的形式是使用二进制包,其中包含已编译的工件和运行时需要的其他静态文件。CMake支持通过名为cpack的命令行工具生成多种类型的包。

下表列出了可用的包生成器:

\begin{table}[H]
	\centering
	\begin{tabular}{|l|l|l|}
		\hline
		\textbf{名称} & \textbf{文件类型} & \textbf{平台}                         \\ \hline
		Archive &
		\begin{tabular}[c]{@{}l@{}}7Z - 7zip - (.7z)\\ TBZ2(.tar.bz2)\\ TGZ(.tar.gz)\\ TXZ(.tar.xz)\\ TZ(.tar.Z)\\ TZST(.tar.zst)\\ ZIP(.zip)\end{tabular} &
		Cross-platform \\ \hline
		Bundle        & Bundle(.bundle)     & macOS                                     \\ \hline
		DEB           & DEB(.deb)           & Linux                                     \\ \hline
		DragNDrop     & DMG(.dmg)           & macOS                                     \\ \hline
		External      & JSON(.json)         & 与外部打包工具集成 \\ \hline
		FreeBSD       & PKG(pkg)            & *BSD,Linux, OSX                           \\ \hline
		IFW           & Binary              & Linux, Windows, macOS                     \\ \hline
		NSIS          & Binary(.exe)        & Windows                                   \\ \hline
		NuGet         & NuGet(.nupkg)       & Windows                                   \\ \hline
		productbuild  & PKG(.pkg)           & macOS                                     \\ \hline
		RPM           & RPM(.rpm)           & Linux                                     \\ \hline
		WIX           & MSI(.msi)           & Windows                                   \\ \hline
	\end{tabular}
\end{table}

大多数生成器都具有丰富的配置。深入研究其细节会超出了本书的范围,所以一定要查看完整的文档,可以在扩展阅读部分找到。相反,我们会关注一般用例。

\begin{tcolorbox}[colback=blue!5!white,colframe=blue!75!black,title=Note]
包生成器不应该与构建系统生成器(Unix Makefiles, Visual Studio等)混淆。
\end{tcolorbox}

要使用CPack,需要使用必要的install()指令正确配置项目的安装,并构建项目。cpack将使用在构建树中生成的生成的cmake\_install.cmake,再基于配置文件(CPackConfig.cmake)准备二进制包。可以手动创建该文件,但使用include(CPack)将实用程序模块包含到项目的列表文件中更容易。其将在项目的构建树中生成配置,并在需要的地方提供所有默认值。

来看看如何扩展示例中的11个组件,使其能够与CPack一起工作:

\begin{lstlisting}[style=styleCMake]
# chapter-11/12-cpack/CMakeLists.txt (fragment)

cmake_minimum_required(VERSION 3.20.0)
project(CPackPackage VERSION 1.2.3 LANGUAGES CXX)
include(GNUInstallDirs)
add_subdirectory(src bin)

install(...)
install(...)
install(...)

set(CPACK_PACKAGE_VENDOR "Rafal Swidzinski")
set(CPACK_PACKAGE_CONTACT "email@example.com")
set(CPACK_PACKAGE_DESCRIPTION "Simple Calculator")
include(CPack)
\end{lstlisting}

代码不难,所以不会过多地讨论(请参考模块文档,可以在扩展阅读部分找到)。CPack模块将从project()指令中推断出一些值:

\begin{itemize}
\item 
CPACK\_PACKAGE\_NAME

\item 
CPACK\_PACKAGE\_VERSION

\item 
CPACK\_PACKAGE\_FILE\_NAME
\end{itemize}

最后一个值将用于生成输出包。其结构如下:

\begin{lstlisting}[style=styleCMake]
$CPACK_PACKAGE_NAME-$CPACK_PACKAGE_VERSION-$CPACK_SYSTEM_NAME
\end{lstlisting}

CPACK\_SYSTEM\_NAME是目标操作系统的名称;例如,Linux或win32。例如,通过在Debian上执行ZIP生成器,CPack将生成一个名为CPackPackage-1.2.3-Linux.zip的文件。

构建项目之后,可以通过在构建树中运行cpack二进制代码来生成实际的包:

\begin{tcblisting}{commandshell={}}
cpack [<options>]
\end{tcblisting}

CPack能够从当前工作目录中的配置文件中读取所有选项,但可以使用命令行覆盖这些设置:

\begin{itemize}
\item 
-G <generators>: 这是一个用分号分隔的包生成器列表。默认值可以在CPackConfig.cmake的CPACK\_GENERATOR变量中指定。

\item 
-C <configs>: 这是一个分号分隔的构建配置(Debug、Release)列表,用于为其生成包(多配置构建系统生成器所需)。

\item 
-D <var>=<value>: 这将使用<value>覆盖CPackConfig.cmake中设置的<var>。

\item 
-{}-config <config-file>: 这是应该使用的配置文件,而不是默认的CPackConfig.cmake。

\item 
-{}-verbose, -V: 提供详细输出。

\item 
-P <packageName>: 覆盖包名称。

\item 
-R <packageVersion>: 覆盖包版本。

\item 
-{}-vendor <vendorName>: 覆盖包供应商。

\item 
-B <packageDirectory>: 指定cpack的输出目录(默认情况下,这将是当前的工作目录)。
\end{itemize}

可以试着为12个包的输出生成包。我们将使用ZIP、7Z和Debian包生成器:

\begin{tcblisting}{commandshell={}}
cpack -G "ZIP;7Z;DEB" -B packages
\end{tcblisting}

应该生成以下包:

\begin{itemize}
\item 
CPackPackage-1.2.3-Linux.7z

\item 
CPackPackage-1.2.3-Linux.deb

\item 
CPackPackage-1.2.3-Linux.zip
\end{itemize}

这种格式的二进制包可以发布到项目网站上,在GitHub中,或者直接发送包到最终用户的邮箱中。












