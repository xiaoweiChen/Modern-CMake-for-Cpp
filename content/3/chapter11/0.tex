
我们的项目已经构建、测试和文档化,是时候向用户发布了。本章主要讨论为此需要采取的最后两个步骤:安装和打包。这些都是高级技术,建立在我们目前所学的基础之上:管理目标及其依赖关系、使用需求、生成器表达式等。

安装允许项目在系统范围内发现和访问。本章中,将介绍如何导出目标,以便另一个项目可以在不安装的情况下使用,以及如何安装项目,以便系统上的程序都可以轻松使用,并学习如何配置项目,以便够自动地将不同的工件类型放在正确的目录中。为了处理更高级的场景,将介绍用于安装文件和目录的低层命令,以及用于执行定制脚本和CMake的低层指令。

接下来,将学习如何设置可重用的CMake包,以便通过从其他项目调用find\_package()来发现它们,将解释如何确保目标及其定义不固定在文件系统上的特定位置。还将讨论如何编写基本和高级配置文件,以及与包相关的版本文件。

然后,为了模块化,将简要介绍组件的概念,包括CMake包和install()指令。所有这些准备工作都是为本章的最后一个方面做准备:使用CPack来生成归档文件、安装程序、打包文件和不同操作系统中所有包管理器都能识别的包。这些可用于携带预构建的工件、可执行文件和库。这是终端用户使用软件最简单的方式。

本章中,我们将讨论以下主题:

\begin{itemize}
\item 
只导出,不安装

\item 
在系统上安装

\item 
创建可重用的包

\item 
定义组件

\item 
使用CPack打包
\end{itemize}









