
In this short chapter, we covered adding Doxygen, the documentation generation tool, to a CMake project, and making it elegant. This process isn't too involved and will greatly improve the flow of information in your solution. Time spent on adding documentation will be a worthwhile investment, especially if you find that you or your teammates have trouble in understanding complex relations in the application.

You may worry that it will be hard to add Doxygen to a bigger project that didn't use documentation generation from the start. The sheer amount of work required to add comments to every function can be overwhelming for developers. Don't strive for immediate completeness: start small, by only filling in a description of elements you touched in your latest commit. Even largely incomplete documentation is better than no documentation at all.

Keep in mind that by generating documentation, you'll enforce its proximity to the actual code: it's way easier to maintain written explanations in sync with the logic if they're both in the same file. Also, realize that as with most programmers, you're probably a very busy person and you will eventually forget some of the small details of your project. Remember: the shortest pencil is longer than the longest memory. Do yourself a favor—write things down and prosper.

In the next chapter, we'll learn how to automate the packaging and installation of our projects with CMake.



