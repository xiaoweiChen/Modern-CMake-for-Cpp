专业人士都明白测试必须自动化。可能几年前就有人跟他们说过了,或者直到吃了苦头才明白。这种实践对于没有经验的开发者来说并不明显:似乎不必要,额外的工作不会带来太多价值。难怪当刚开始编写代码时,会避免编写复杂的解决方案和贡献大量的代码库。最有可能的情况是,他们喜欢自己是项目的唯一开发者。这些早期的项目几乎不需要超过几个月的时间来完成,所以几乎没有机会看到代码在历史长河中是如何腐烂的。

这些因素让人感觉编写测试就是在浪费时间和精力。编程学徒可能会对自己说,每次执行“构建-运行”例程时,实际上都测试了代码。毕竟,他们已经手动确认了代码是否正常工作,并完成了预期的工作。终于到了进行下一个任务的时候了,对吧?

自动化测试保证了新的更改不会意外地破坏程序。本章中,将了解测试得重要性,以及如何使用CTest(CMake附带的一个工具)来协调测试的执行。CTest能够查询可用的测试、过滤、打乱、重复和时间限制。我们将探讨如何使用这些特性、控制CTest的输出,以及如何处理失败的测试。

接下来,将调整我们的项目结构,以支持测试并创建自己的测试运行器。在讨论了基本原则之后,将继续添加流行的测试框架:Catch2和带有mock库的GoogleTest。最后,将介绍使用LCOV的详细测试覆盖率报告。

本章中,我们将讨论以下主题:

\begin{itemize}
\item 
为什么自动化测试值得这么麻烦?

\item 
使用CTest来标准化CMake中的测试

\item 
为CTest创建最基本的单元测试

\item 
单元测试框架

\item 
生成测试覆盖报告
\end{itemize}






















