该指令提供与文件相关的各种操作:读取、传输、锁定和归档,还提供了检查文件系统和对表示路径的字符串的操作的模式。

完整的细节可以在在线文档中找到:

\url{https://cmake.org/cmake/help/latest/command/file.html}

\hspace*{\fill} \\ %插入空行
\noindent
\textbf{读取}

有以下几种模式:

\begin{itemize}
\item 
file(READ <filename> <out> [OFFSET <o>] [LIMIT <max>] [HEX]) 从<filename>读取文件到<out>变量。读取可选地从偏移量<o>开始,并遵循<max>字节的可选限制。HEX标志指定输出应该转换为十六进制表示。

\item 
file(STRINGS <filename> <out>) 从<filename>处的文件读取字符串到<out>变量。

\item 
file(<algorithm> <filename> <out>) 从<filename>的文件计算<algorithm>哈希,并将结果存储在<out>变量中。可用的算法与string()哈希算法相同。

\item 
file(TIMESTAMP <filename> <out> [<format>]) 生成文件名<filename>的文件时间戳的字符串表示形式,并将其存储在<out>变量中,可选地接受<format>字符串。

\item 
file(GET\_RUNTIME\_DEPENDENCIES [...]) 获取指定文件的运行时依赖项。这是一个高级命令,只能在install(CODE)或install(SCRIPT)中使用。
\end{itemize}


\hspace*{\fill} \\ %插入空行
\noindent
\textbf{写入}

有以下几种模式:

\begin{itemize}
\item 
file(\{WRITE | APPEND\} <filename> <content>...) 将所有<content>参数写入或追加到<filename>的文件。若提供的系统路径不存在,将递归地创建它。

\item 
file(\{TOUCH | TOUCH\_NOCREATE\} [<filename>...]) 更新<filename>的时间戳。若文件不存在,将只在TOUCH模式下创建。

\item 
file(GENERATE OUTPUT <output-file> [...]) 是一种高级模式,它为当前CMake Generator的每个构建配置生成一个输出文件。

\item 
file(CONFIGURE OUTPUT <output-file> CONTENT <content> [...]) 像GENERATE\_OUTPUT一样工作,也可以通过用值替换变量占位符来配置生成的文件。
\end{itemize}


\hspace*{\fill} \\ %插入空行
\noindent
\textbf{文件系统}

有以下几种模式:

\begin{itemize}
\item 
file(\{GLOB | GLOB\_RECURSE\} <out> [...] [<globbingexpression>...]) 生成匹配<globbingexpression>的文件列表,并将其存储在<out>变量中。GLOB\_RECURSE模式也将扫描嵌套目录。

\item 
file(RENAME <oldname> <newname>) 将文件从<oldname>移动到<newname>。

\item 
file({REMOVE | REMOVE\_RECURSE } [<files>...]) 删除<files>。REMOVE\_RECURSE会删除目录。

\item 
file(MAKE\_DIRECTORY [<dir>...]) 创建目录

\item 
file(COPY <file>... DESTINATION <dir> [...]) 将文件复制到
<dir>目标。提供筛选、设置权限、软链接等选项。

\item 
file(SIZE <filename> <out>) 读取<filename>的字节大小,并将其存储在<out>变量中。

\item 
file(READ\_SYMLINK <linkname> <out>) 读取<linkname>符号链接的目标路径,并将其存储在<out>变量中。

\item
file(CREATE\_LINK <original> <linkname> [...]) 在<linkname>处创建一个指向<original>的软链接。

\item 
file(\{CHMOD|CHMOD\_RECURSE\} <files>... <directories>... PERMISSIONS <permissions>... [...]) 设置文件和目录的权限。
\end{itemize}

\hspace*{\fill} \\ %插入空行
\noindent
\textbf{路径转换}

有以下几种模式:

\begin{itemize}
\item 
file(REAL\_PATH <path> <out> [BASE\_DIRECTORY <dir>]) 从相对路径计算绝对路径并将其存储在<out>变量中。可选地接受
<dir>基目录。从CMake 3.19开始可用。

\item 
file(RELATIVE\_PATH <out> <directory> <file>) 计算相对于<directory>的<file>路径,并将其存储在<out>变量中。

\item 
file(\{TO\_CMAKE\_PATH | TO\_NATIVE\_PATH\} <path> <out>) 将转换为CMake路径(用正斜杠分隔的目录)到平台的本机路径并返回,结果存储在<out>变量中。
\end{itemize}

\hspace*{\fill} \\ %插入空行
\noindent
\textbf{传输}

有以下几种模式:

\begin{itemize}
\item 
file(DOWNLOAD <url> [<path>] [...]) 从<url>下载文件并将其存储在path中。

\item 
file(UPLOAD <file> <url> [...]) 上传<file>到URL。
\end{itemize}


\hspace*{\fill} \\ %插入空行
\noindent
\textbf{锁定}

锁定模式在<path>资源上放置一个建议锁:

\begin{lstlisting}[style=styleCMake]
file(LOCK <path> [DIRECTORY] [RELEASE]
	[GUARD <FUNCTION|FILE|PROCESS>]
	[RESULT_VARIABLE <out>]
	[TIMEOUT <seconds>])
\end{lstlisting}

该锁可以选择作用域为FUNCTION、FILE或PROCESS,并以<seconds>的超时进行限制。要释放锁,需要RELEASE关键字,结果将存储在<out>变量中。

\hspace*{\fill} \\ %插入空行
\noindent
\textbf{归档}

归档文件的创建具有以下签名:

\begin{lstlisting}[style=styleCMake]
file(ARCHIVE_CREATE OUTPUT <destination> PATHS <source>...
	[FORMAT <format>]
	[COMPRESSION <type> [COMPRESSION_LEVEL <level>]]
	[MTIME <mtime>] [VERBOSE])
\end{lstlisting}

它在<destination>路径上创建一个存档文件,其中包含支持的格式之一的<source>文件:7zip、gnutar、pax、paxr、raw或zip (paxr是默认格式)。若所选格式支持压缩级别,则可以将其提供为一个单位数整数0-9,其中0为默认值。

提取模式具有以下签名:

\begin{lstlisting}[style=styleCMake]
file(ARCHIVE_EXTRACT INPUT <archive> [DESTINATION <dir>]
	[PATTERNS <patterns>...] [LIST_ONLY] [VERBOSE])
\end{lstlisting}

它将匹配可选<patterns>值的文件从<archive>提取到目标<dir>。若有LIST\_ONLY关键字,则不会提取文件,而是只列出文件。



