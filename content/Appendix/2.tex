This command provides basic operations on lists: reading, searching, modification, and ordering. Some modes will change list (mutate the original value). Be sure to copy the original value if you'll need it later.

Full details can be found in the online documentation: 

\url{https://cmake.org/cmake/help/latest/command/list.html}

\hspace*{\fill} \\ %插入空行
\noindent
\textbf{Reading}

The following modes are available:

\begin{itemize}
\item 
list(LENGTH <list> <out>) counts the elements in the <list> variable and stores the result in the <out> variable.

\item 
list(GET <list> <index>... <out>) copies the <list> elements specified with the list of <index> indexes to the <out> variable.

\item 
list(JOIN <list> <glue> <out>) interleaves <list> elements with the <glue> delimiter and stores the resulting string in the <out> variable.

\item 
list(SUBLIST <list> <begin> <length> <out>) works like the GET mode, but operates on range instead of explicit indexes. If <length> is -1, elements from <begin> index to the end of the list provided in the <list> variable will be returned.
\end{itemize}

\hspace*{\fill} \\ %插入空行
\noindent
\textbf{Searching}

This mode simply finds the index of the <needle> element in the <list> variable and stores the result in the <out> variable (or -1 if the element wasn't found):

\begin{lstlisting}[style=styleCMake]
list(FIND <list> <needle> <out>)
\end{lstlisting}

\hspace*{\fill} \\ %插入空行
\noindent
\textbf{Modification}

The following modes are available:

\begin{itemize}
\item 
list(APPEND <list> <element>...) adds one or more <element> value to the end of the <list> variable.

\item 
list(PREPEND <list> [<element>...]) works like APPEND, but adds elements to the beginning of the <list> variable.

\item 
list(FILTER <list> {INCLUDE | EXCLUDE} REGEX <pattern>) filters the <list> variable to INCLUDE or EXCLUDE the elements matching the <pattern> value.

\item 
list(INSERT <list> <index> [<element>...]) adds one or more <element> values to the <list> variable at the given <index>.

\item 
list(POP\_BACK <list> [<out>...]) removes an element from the end of the <list> variable and stores it in the optional <out> variable. If multiple <out> variables were provided, more elements will be removed to fill them.

\item 
list(POP\_FRONT <list> [<out>...]) works like POP\_BACK but removes an element from the beginning of the <list> variable.

\item 
list(REMOVE\_ITEM <list> <value>...) shorthand for FILTER EXCLUDE, but without the support of regular expressions.

\item 
list(REMOVE\_AT <list> <index>...) removes elements from <list> at a specific <index>.

\item 
list(REMOVE\_DUPLICATES <list>) removes duplicates from <list>.

\item 
list(TRANSFORM <list> <action> [<selector>] [OUTPUT\_VARIABLE <out>]) applies a specific transformation to the <list> elements.

By default, the action is applied to all elements, but we may limit the effect by adding a <selector>. Provided list will be mutated (changed in place) unless the OUTPUT\_VARIABLE keyword was provided, in which case, the result is stored in the <out> variable.

The following selectors are available: AT <index>, FOR <start> <stop> [<step>], and REGEX <pattern>. Actions include APPEND <string>, PREPEND <string>, TOLOWER, TOUPPER, STRIP, GENEX\_STRIP, and REPLACE <pattern> <expression>. They work exactly like the string() modes with the same name.
\end{itemize}


\hspace*{\fill} \\ %插入空行
\noindent
\textbf{Ordering}

The following modes are available:

\begin{itemize}
\item 
list(REVERSE <list>) simply reverses the order of <list>.

\item 
list(SORT <list>) sorts the list alphabetically. Refer to the online manual for more advanced options.
\end{itemize}

